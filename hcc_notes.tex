% Options for packages loaded elsewhere
\PassOptionsToPackage{unicode}{hyperref}
\PassOptionsToPackage{hyphens}{url}
%
\documentclass[
]{book}
\title{Reading Notes}
\author{HCC}
\date{2023-04-16}

\usepackage{amsmath,amssymb}
\usepackage{lmodern}
\usepackage{iftex}
\ifPDFTeX
  \usepackage[T1]{fontenc}
  \usepackage[utf8]{inputenc}
  \usepackage{textcomp} % provide euro and other symbols
\else % if luatex or xetex
  \usepackage{unicode-math}
  \defaultfontfeatures{Scale=MatchLowercase}
  \defaultfontfeatures[\rmfamily]{Ligatures=TeX,Scale=1}
\fi
% Use upquote if available, for straight quotes in verbatim environments
\IfFileExists{upquote.sty}{\usepackage{upquote}}{}
\IfFileExists{microtype.sty}{% use microtype if available
  \usepackage[]{microtype}
  \UseMicrotypeSet[protrusion]{basicmath} % disable protrusion for tt fonts
}{}
\makeatletter
\@ifundefined{KOMAClassName}{% if non-KOMA class
  \IfFileExists{parskip.sty}{%
    \usepackage{parskip}
  }{% else
    \setlength{\parindent}{0pt}
    \setlength{\parskip}{6pt plus 2pt minus 1pt}}
}{% if KOMA class
  \KOMAoptions{parskip=half}}
\makeatother
\usepackage{xcolor}
\IfFileExists{xurl.sty}{\usepackage{xurl}}{} % add URL line breaks if available
\IfFileExists{bookmark.sty}{\usepackage{bookmark}}{\usepackage{hyperref}}
\hypersetup{
  pdftitle={Reading Notes},
  pdfauthor={HCC},
  hidelinks,
  pdfcreator={LaTeX via pandoc}}
\urlstyle{same} % disable monospaced font for URLs
\usepackage{longtable,booktabs,array}
\usepackage{calc} % for calculating minipage widths
% Correct order of tables after \paragraph or \subparagraph
\usepackage{etoolbox}
\makeatletter
\patchcmd\longtable{\par}{\if@noskipsec\mbox{}\fi\par}{}{}
\makeatother
% Allow footnotes in longtable head/foot
\IfFileExists{footnotehyper.sty}{\usepackage{footnotehyper}}{\usepackage{footnote}}
\makesavenoteenv{longtable}
\usepackage{graphicx}
\makeatletter
\def\maxwidth{\ifdim\Gin@nat@width>\linewidth\linewidth\else\Gin@nat@width\fi}
\def\maxheight{\ifdim\Gin@nat@height>\textheight\textheight\else\Gin@nat@height\fi}
\makeatother
% Scale images if necessary, so that they will not overflow the page
% margins by default, and it is still possible to overwrite the defaults
% using explicit options in \includegraphics[width, height, ...]{}
\setkeys{Gin}{width=\maxwidth,height=\maxheight,keepaspectratio}
% Set default figure placement to htbp
\makeatletter
\def\fps@figure{htbp}
\makeatother
\setlength{\emergencystretch}{3em} % prevent overfull lines
\providecommand{\tightlist}{%
  \setlength{\itemsep}{0pt}\setlength{\parskip}{0pt}}
\setcounter{secnumdepth}{5}
\newlength{\cslhangindent}
\setlength{\cslhangindent}{1.5em}
\newlength{\csllabelwidth}
\setlength{\csllabelwidth}{3em}
\newlength{\cslentryspacingunit} % times entry-spacing
\setlength{\cslentryspacingunit}{\parskip}
\newenvironment{CSLReferences}[2] % #1 hanging-ident, #2 entry spacing
 {% don't indent paragraphs
  \setlength{\parindent}{0pt}
  % turn on hanging indent if param 1 is 1
  \ifodd #1
  \let\oldpar\par
  \def\par{\hangindent=\cslhangindent\oldpar}
  \fi
  % set entry spacing
  \setlength{\parskip}{#2\cslentryspacingunit}
 }%
 {}
\usepackage{calc}
\newcommand{\CSLBlock}[1]{#1\hfill\break}
\newcommand{\CSLLeftMargin}[1]{\parbox[t]{\csllabelwidth}{#1}}
\newcommand{\CSLRightInline}[1]{\parbox[t]{\linewidth - \csllabelwidth}{#1}\break}
\newcommand{\CSLIndent}[1]{\hspace{\cslhangindent}#1}
\ifLuaTeX
  \usepackage{selnolig}  % disable illegal ligatures
\fi

\begin{document}
\maketitle

{
\setcounter{tocdepth}{1}
\tableofcontents
}
\hypertarget{current-reading-list}{%
\chapter*{Current Reading List}\label{current-reading-list}}
\addcontentsline{toc}{chapter}{Current Reading List}

\begin{enumerate}
\def\labelenumi{\arabic{enumi}.}
\tightlist
\item
  (\protect\hyperlink{ref-becker2006string}{Becker, Becker, and Schwarz 2006})
\item
  (\protect\hyperlink{ref-gillioz2022conformal}{Gillioz 2022})
\end{enumerate}

\begin{center}\rule{0.5\linewidth}{0.5pt}\end{center}

\begin{longtable}[]{@{}l@{}}
\toprule
\endhead
output: pandoc\_args: {[}``--number-offset=0''{]} \\
\bottomrule
\end{longtable}

\hypertarget{becker-becker-schwarz}{%
\chapter*{Becker-Becker-Schwarz}\label{becker-becker-schwarz}}
\addcontentsline{toc}{chapter}{Becker-Becker-Schwarz}

Reading Ch1 of (\protect\hyperlink{ref-becker2006string}{Becker, Becker, and Schwarz 2006})

\hypertarget{why}{%
\section{Why?}\label{why}}

One important feature is the ability to generate GR via its dynamics.

\begin{itemize}
\tightlist
\item
  ``QFT \(\rightarrow\) ST'' \(\Rightarrow\) ``World-lines\footnote{Maybe we need to familar ourselves with the word-line description first:
    \href{https://arxiv.org/abs/hep-ph/9205205}{Strassler, 1992, Field Theory Without Feynman Diagrams: One-Loop Effective Actions}; \href{https://arxiv.org/abs/1912.10004}{Schubert, 2019, Quantum mechanical path integrals in the first quantised approach to quantum field theory}; \href{https://arxiv.org/abs/hep-th/0101036}{Schubert, Perturbative Quantum Field Theory in the String-Inspired Formalism}; \href{https://physics.stackexchange.com/questions/522717/worldline-formulation-of-qft}{stackexchange: Worldline formulation of QFT}} \(\rightarrow\) World-sheet''

  \begin{itemize}
  \tightlist
  \item
    QFT interaction: local singularties on the junctions of different lines ending-afar
  \item
    ST interation: topology on the whole sheet

    \begin{itemize}
    \tightlist
    \item
      Hence,the UV divergence ``must''(if it can be done consistently at all) be resolved in the string picture because above the string scale the smooth nature of the string dynamics resolves the sigularity.
    \end{itemize}
  \end{itemize}
\end{itemize}

\hypertarget{two-formulations}{%
\section{Two formulations}\label{two-formulations}}

\hypertarget{nambu-go-action}{%
\subsection{Nambu-Go Action:}\label{nambu-go-action}}

\[
S_{p} = -T_{p}V
\]
\#\#\# String Sigma Model:
\[
S_{\sigma} = - \frac{T}{2} \int \sqrt{-h} h^{\alpha\beta}\eta_{\mu\nu}
\partial_{\alpha}X^{\mu}
\partial_{\beta}X^{\nu},
\]

where
- \(h^{\alpha\beta}(\sigma, \tau)\) the auxiliary\footnote{the auxiliary here is very apopos: we will gauge its nature later.} world-sheet metric,
- \(h=\text{det} \,\, h_{\alpha\beta}\),
- \(h^{\alpha\beta}\) is the inverse of \(h_{\alpha\beta}\), as \(h^{\alpha\beta} = [h^{-1}]^{\alpha\beta}\),
- \(X^{\mu}(\sigma, \tau)\), the embedding of the world-sheet into the ambient space-time, aka
- \((\sigma, \tau) \in \mathcal{M}^2 \rightarrow X^{\mu} \in \mathcal{R}^\mathcal{D}\)

\hypertarget{the-equvalence-of-the-two}{%
\section{The equvalence of the two}\label{the-equvalence-of-the-two}}

Classically, we can use \(S_\sigma\) to find the EOM of \(h_{\alpha\beta}\), and it can be shown that \(S_\sigma\) localized at those \(h\)-locale will reccover the the Nambu-Goto on \(X^\mu\)

However, quantum mechanically, we have the \(\hbar\) to deal with, as well as the gauge-fixing and what not =\textgreater{} (Spoiler alert!!) All in all, we will find that there is a conformal anomaly unless \(mathcal{D}=26\) for bosonic strings, and \(\mathcal{D}=10\) for superstrings

\hypertarget{first-superstring-revolution-1984}{%
\subsection{First Superstring revolution, 1984}\label{first-superstring-revolution-1984}}

The discovery: Q-consistency of \(\mathcal{N}=1\) 10-dim SST requires a local Yang-Mills gauge symmetry on either \(SO(32)\) or \(E_8\times E_8\) (cancelation of q-anomalies, see Ch5, also with \(U(1)^{496}\) and \(E_8 \times U(1)^{248}\))

\begin{itemize}
\tightlist
\item
  \(\mathcal{N}=2\)

  \begin{itemize}
  \tightlist
  \item
    type II: we have both left- and right-moving modes, and we get to choose the handedness on each, either the opposite (IIA) or the same (IIB);
  \end{itemize}
\item
  \(\mathcal{N}=1\)

  \begin{itemize}
  \tightlist
  \item
    type I: for IIB above, we have the left-right symmetry, and we get to mode this symmetery out (orbifold projection), wit the remaining content being type I;
  \item
    2 ``Heterotic''s: left being left-moving \(\mathcal(D)=26\) bosonic strings, right being right-moving \(\mathcal(D)=11\) superstrings! The 16 extra dimension on the left must be some tori to have a final consistent (q-anomaly-free) formulation: Two such tori, each corrresponds to \(SO(32)\) or \(E_8\times E_8\).
  \end{itemize}
\end{itemize}

\hypertarget{second-superstring-revolutionlate-1980s-to-1990s}{%
\subsection{Second Superstring revolution,late 1980s to 1990s}\label{second-superstring-revolutionlate-1980s-to-1990s}}

Too many ultimate theories now\ldots however, T-duality is found to related the twp type-Is and type-IIs! With the nonperturbative S-duality and the opening up of another dimension at strong coupling, we have the final united theory!

T-duality: two different geometries in the extra dimension are physically q-equivalent!
- Example: circle with radius \(R\) is the same as circle with radius \(\frac{l^2_s}{R}\)

S-duality: \(g_s \leftrightarrow \frac{1}{g_s}\)

\begin{itemize}
\tightlist
\item
  type-IIB with itself
\item
  type-IA with \(\text{Heterotic}|_{SO(32)}\)
\item
  How about type-IIA and \(\text{Heterotic}|_{E_8 \times E_8}\)?

  \begin{itemize}
  \tightlist
  \item
    they grow an extra dimension of the size of \(g_s l_s\)!

    \begin{itemize}
    \tightlist
    \item
      A circle for type-IIA;
    \item
      A line interval for \(\text{Heterotic}|_{E_8 \times E_8}\) )
    \end{itemize}
  \end{itemize}
\end{itemize}

Matrix theory

\begin{itemize}
\tightlist
\item
  A dual description of M-theory in flat 11-dimensional space-time
\item
  QM of N\(\times\)N matrices in the large N limit
\end{itemize}

F-theory

\begin{itemize}
\tightlist
\item
  We have S-duality opening up type-IIA and \(\text{Heterotic}|_{E_8 \times E_8}\) to M-theory \ldots{} how about others?
\item
  type-IIB has a non-perturbative \(SL(2,\mathbb{Z})\)-symmetry -- which is naturally associated with a fictitious torus (the modular group of a torus) + type-IIB also has a complex scalar field \(\tau\) transforming under \(SL(2,\mathbb{Z})\)

  \begin{itemize}
  \tightlist
  \item
    We can see this symmetry geometrically, having an auxiliary two-torus \(T^2\) with complex structure \(\tau\).
  \end{itemize}
\end{itemize}

\begin{longtable}[]{@{}l@{}}
\toprule
\endhead
output: pandoc\_args: {[}``--number-offset=0''{]} \\
\bottomrule
\end{longtable}

\hypertarget{gillioz-cft-v00.00}{%
\chapter*{Gillioz CFT v00.00}\label{gillioz-cft-v00.00}}
\addcontentsline{toc}{chapter}{Gillioz CFT v00.00}

Reading (\protect\hyperlink{ref-gillioz2022conformal}{Gillioz 2022})

\hypertarget{preface}{%
\section*{Preface}\label{preface}}
\addcontentsline{toc}{section}{Preface}

\begin{itemize}
\tightlist
\item
  Two more symmetries: scale symmetry, and special conformal symmetry

  \begin{itemize}
  \tightlist
  \item
    Modern conformal bootstrap!
  \item
    Unitarity bounds derived from the spectral representation!
  \item
    Appearance of UV and IR divergences!
  \end{itemize}
\end{itemize}

\hypertarget{ch1-intro}{%
\section{Ch1: Intro}\label{ch1-intro}}

\begin{itemize}
\tightlist
\item
  New Definition:
  \begin{align}
  \mathrm{CFT}  =& \mathrm{RQFT}  \\
        &+ \mathrm{Scale Sym}\\
       &+ \mathrm{Special-Conformal Sym}
  \end{align}

  \begin{itemize}
  \tightlist
  \item
    RQFT: flat-Minkowski, Wightman axioms, and all that.
  \item
    Scale Sym: RG, no-mass gap, and all that
  \item
    Special-Conformal Sym: conformal correlation functions, OPE(!), conformal bootstrap(!!)
  \end{itemize}
\end{itemize}

\hypertarget{ch2-classical-cft}{%
\section{Ch2: C(lassical)-CFT}\label{ch2-classical-cft}}

\begin{itemize}
\tightlist
\item
  The physical symmetry is a special transformation enjoyed by a physical system, and in the most general term it can be set up tentatively (without putting requirement at all) as:
  \[
  x^\mu \rightarrow x'^\mu = x^\mu + \varepsilon^\mu(x) + \mathcal{O}(\varepsilon^2)
  \]
\item
  The Physical distance, aka \(ds = g_{\mu\nu}(x) dx^\mu dx^\nu\)

  \begin{itemize}
  \tightlist
  \item
    Either Eucliean \(\delta_{\mu\nu}\) or Minskowski \(\eta_{\mu\nu}\)
  \item
    Distance being physical, aka invariant under valid physically sensible transformation:
    \[
    g'_{\mu\nu}(x') dx'^\mu dx'^\nu = g_{\alpha\beta}(x) dx^\alpha dx^\beta
    \]
  \item
    Assuming the metric is flat, aka \(\partial_{\alpha} g_{\mu\nu} \equiv 0\), then
  \end{itemize}
\end{itemize}

\begin{align}
  g'_{\mu\nu}(x')  =&  
  g_{\alpha\beta}
  \frac{\partial x^\alpha}{\partial x'^\mu}
  \frac{\partial x^\beta}{\partial x'^\nu} \\
  =& g_{\alpha\beta} 
  (\delta^\alpha_\mu - \partial_\mu \varepsilon^\alpha)
  (\delta^\beta_\nu - \partial_\nu \varepsilon^\beta) 
  + \mathcal{O}(\varepsilon^2) \\
  =&  g_{\mu\nu} - \partial_\mu \varepsilon_\nu - \partial_\nu \varepsilon_\mu 
  + \mathcal{O}(\varepsilon^2) 
\end{align}

\begin{itemize}
\tightlist
\item
  Therefore, in a flat metric, for the distance to be preserved, the would-be good physical symmetry cannot be arbitrary in that it needs to obey the constrain of \(\partial_\mu \varepsilon_\nu + \partial_\nu \varepsilon_\mu \overset{!}{=} 0\).

  \begin{itemize}
  \tightlist
  \item
    The most general solution: \(\varepsilon_\mu = a^\mu + \omega_{\mu\nu}(x) x^\nu\), with \(a^\mu\) be a constant, and \(\omega_{\mu\nu}\) be anti-symmetric tensor.

    \begin{itemize}
    \tightlist
    \item
      \([a^\mu|x^\mu\rightarrow{x^\mu+a^\mu}]\), the translation group!
    \item
      \([\omega_{\mu\nu}|x^\mu\rightarrow{x^\mu+\omega_{\mu\nu}x^\nu}]\): the rotation group!

      \begin{itemize}
      \tightlist
      \item
        \([a^\mu,\omega_{\mu\nu}]\), the Poincare group!
      \end{itemize}
    \end{itemize}
  \end{itemize}
\end{itemize}

\hypertarget{reference}{%
\section{Reference}\label{reference}}

\hypertarget{refs}{}
\begin{CSLReferences}{1}{0}
\leavevmode\vadjust pre{\hypertarget{ref-becker2006string}{}}%
Becker, Katrin, Melanie Becker, and John H Schwarz. 2006. \emph{String Theory and m-Theory: A Modern Introduction}. Cambridge university press.

\leavevmode\vadjust pre{\hypertarget{ref-gillioz2022conformal}{}}%
Gillioz, Marc. 2022. {``Conformal Field Theory for Particle Physicists.''} \emph{arXiv Preprint arXiv:2207.09474}.

\end{CSLReferences}

\end{document}
